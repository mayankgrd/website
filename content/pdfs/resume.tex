% LaTeX file for resume 
% This file uses the resume document class (res.cls)

\documentclass[margin]{res} 
% the margin option causes section titles to appear to the left of body text 
\usepackage[hidelinks]{hyperref}
\hypersetup{colorlinks=false}
\textwidth=5.2in % increase textwidth to get smaller right margin
%\usepackage{helvetica} % uses helvetica postscript font (download helvetica.sty)
%\usepackage{newcent}   % uses new century schoolbook postscript font 

\begin{document} 
 
\name{\large{Mayank Kumar}\\[12pt]} % the \\[12pt] adds a blank line after name

\address{Rice University \\ ECE Department, MS-380 \\ 6100 Main Street  \\ Houston, TX 77005         }
        
\address{{} \\ \emph{Phone:}     832-593-1893 \\ \emph{Email:}   mk28@rice.edu \\ \emph{Website:}
\url{www.ece.rice.edu/~mk28/}}


 
\begin{resume} 
 
\section{Objective} 
To be a leader in technology which touches human lives  

\section{Education} 
PhD candidate, Electrical and Computer Engineering, Rice University \hfill{2014 - Present} \\
 \emph{Advised by Dr. Ashutosh Sabharwal} \\
MS in Electrical and Computer Engineering, Rice University \hfill{Aug, 2014}  \\
 \emph{GPA $4.04/4.00$} \\
B.Tech in Electrical Engineering, IIT, Delhi \hfill{May, 2010}  \\
 \emph{GPA $8.96 / 10.00$, Department Rank 3}   \\

\vspace{-5mm}
\section{Scholastic \\ Achievements}
NSF Awards for young professionals contributing to smart and connected health (2016) 
Texas Instruments Graduate Student Fellowship (2015-Present) \\
“Audience Choice Award”, Rice 90 Second Thesis Competition 2014  \\
“Best Graduate Student Poster”, Rice ECE Affiliates Day 2014 \\
NASA Space Health Challenge 2014 (2nd Prize)  \\
Best B.Tech Project Award in IIT Delhi, 2010 \\
Yahoo HackU Award, 2009  by Yahoo R\&D \\
Indian National Physics Olympiad, 2006

\section{MS Research}
Robust estimations of Photoplethysmograms using a camera \hfill{Spring 2014 }\\
 \begin{itemize} \itemsep -2pt  % reduce space between items
 \item Developed a new algorithm (distancePPG) for monitoring vital signs (pulse rate, pulse rate variability, breathing rate) using a person's video 
 \item Improved performance of existing methods to make it work for all skin tones, under varied lighting conditions and in different motion scenarios 
 \end{itemize}

\section{Publications}
[J1] Mayank Kumar, Ashok Veeraraghavan, and Ashutosh Sabharwal, "DistancePPG: Robust non-contact vital signs monitoring using a camera," Biomed. Opt. Express 6, 1565-1588 (2015)

[C1] Mayank Kumar, James Suliburk, Ashok Veeraraghavan and Ashutosh Sabharwal, "PulseCam: High-resolution blood perfusion imaging using a camera and a pulse oximeter," 2016 38th Annual International Conference of the IEEE Engineering in Medicine and Biology Society (EMBC), Orlando, FL, 2016, pp. 3904-3909.

[C2] Peter Washington, Mayank Kumar, Anant Tibrewal, and Ashutosh Sabharwal, 'ScaleMed: A Methodology for Iterative mHealth Clinical Trials' IEEE Healthcom 2015 - SSH 2015.

[C3] M. Chowdhary, CSR Technology, USA; M. Sharma, A. Kumar, IIT, India; S. Dayal, CSR Technology, India; M. Kumar, IIT, India. Robust Attitude Estimation for Indoor Pedestrian Navigation using MEMS Sensors. ION GNSS 2012 

[C4] Dhruv Jain, Himanshu Gupta, Deeksha Gautam, Mayank Kumar, Vinay Ribeiro, Manish Sharma. Whitespace Network for Vehicular Communication. COMSNETS 2013

\section{Patents}
[P1] Camera-based photoplethysmogram estimation (US Utility Patent Application, Nov 2015) 

[P2] High resolution blood perfusion imaging using a camera and a pulse oximeter (US Provisional Patent, April 2016)

[P3] A system and apparatus for Auditory Evoked Potential (AEP) data acquisition for hearing screening (India Patent, 2011)

\section{Experience}
  {\bf Innovator-in-Residence,} Gauss Surgical Inc., Los Altos, CA.  \hfill Summer  2015
  \begin{itemize} \itemsep -2pt
  \item Explored the prospect of productizing non-contact vital sign monitoring, developed minimum viable prototype. 
  
  \end{itemize}
  
  
 {\bf Teaching Assistant,} Rice University, ECE Dept.  \hfill Fall  2014
 \begin{itemize} \itemsep -2pt  % reduce space between items
 \item Conduct weekly concept review sessions for ELEC-$241$, Fundamental of Electric Engineering 

 \end{itemize}

 {\bf Corporate R\&D Intern,} Qualcomm, San Diego, CA \hfill Summer  2013
 \begin{itemize} \itemsep -2pt  % reduce space between items
 \item Developed new algorithm for non-linear interference cancellation (NLIC) in $4$G communication systems. 
 \item Proposed and implemented new ideas to reduce convergence time and implementation cost of developed algorithm 
 \end{itemize}

{\bf Algorithm Developer,} Stanford India Bio-design, AIIMS New Delhi \hfill Fall 2011
\begin{itemize} \itemsep -2pt %reduce space between items
\item Devised novel algorithm for detecting weak ($100$~nV) Auditory Brainstem Response (ABR) signal in presence of $30$~dB high EM noise
\item Patented the algorithm (and prototype) (India Patent)
 
\end{itemize}


{\bf Algorithm Developer,} CSR plc, Noida,  \hfill
Spring 2011
\begin{itemize} \itemsep -2pt
 \item Developed error model using Extended Kalman filter (EKF) for a mobile phone based pedestrian navigation system
 \item Benchmarked performance of navigation algorithm and suggested scope of improvements

\end{itemize}
 
{\bf Engineering Trainee,} Texas Instruments, Bangalore  \hfill
Summer 2009
\begin{itemize} \itemsep -2pt
 \item Developed an application to measure performance of GPS receivers during field trials in absence of ground truth data 


\end{itemize}



\section{Leadership   Experience} 
{\bf Co-founder,} Yantrr Electronic Systems (YES) Pvt. Ltd.     \hfill         2010-2015
\begin{itemize} \itemsep -2pt
\item Developed the cloud architecture for Yantrr M2M device cloud
\item Shaped the strategy for YES to be a leader in Industrial machine-to-machine communications
\end{itemize}

\section{Technical Skills}
{\bf Programming Language}: Python, C/C++, MATLAB, VHDL \\
{\bf Software Libraries}: OpenCV (Computer Vision), Scikit Learn (Machine Learning), TensorFlow (deep-learning), OpeGL (Computer Graphics) \\
{\bf Hardware Platform}: Arduino, BeagleBone, C5515 TI DSP, DM365 DaVinci, Xilinx Virtex IV 
\section{References*} 
Dr. Ashutosh Sabharwal (Prof. Rice, ECE), Dr. Ashok Veeraraghavan (Asst. Prof. Rice, ECE), James W. Suliburk, MD, FACS (Asst. Prof. of Surgery, Baylor College of Medicine), Siddharth Satish (Founder and CEO, Gauss Surgical) \\
\qquad \emph{* All references are made available upon request}

\end{resume} 
\end{document} 



